%
% PPUA 5262 Final
%
\documentclass[12pt]{article}

%
% Packages
%
\usepackage{amsmath}
\usepackage{amssymb}
\usepackage{amsthm}
\usepackage{clrscode3e}
\usepackage{enumerate}
\usepackage{graphicx}
\usepackage[table]{xcolor}% http://ctan.org/pkg/xcolor
\usepackage{caption}
\usepackage{subcaption}


%
% Package settings
%

%% begin.rcode setup, include=FALSE
% library(knitr)
%
% opts_chunk$set(fig.path='figure/latex-', cache.path='cache/latex-')
%% end.rcode

%
% Document Settings
%
\setlength{\parskip}{1pc}
\setlength{\parindent}{0pt}
\setlength{\topmargin}{-3pc}
\setlength{\textheight}{9.5in}
\setlength{\oddsidemargin}{0pc}
\setlength{\evensidemargin}{0pc}
\setlength{\textwidth}{6.5in}


\title{Quick, Draw! Image Classification}
\author{Group 3}

%
% Image directory location
%
\graphicspath{ {./imgs/} }

\begin{document}

\maketitle

\section{Introduction}

Many recent industry innovations have included neural networks. An
important research area within neural networks has been image classification.
One of the best techniques to classify images is Convolutional Neural
Networks (CNN). We have sought to better understand industry best practices
by exploring different image classification techniques with neural networks.
Our exploration starts by manually performing image classification and
noting the level of human accuracy. We then use a fully connected neural
network to perform image classification before implementing a CNN. We
implemented options for tuning and debugging the neural network while
implementing each method during our process. We were able to develop an
intuition for image classification best practices using neural networks
after our research.

Many supervised machine learning methods often differ from those employed 
when constructing neural networks. These supervised machine learning
methods, such as logistic regression or decision tree classification,
require feature engineering to successfully classify an image. Models
requiring feature engineering rely on considerable effort by subject matter
experts to achieve reasonable accuracy. Neural networks allow for an
approach which does not require feature engineering to achieve similar or
better accuracy. Industry engineering efforts regarding feature engineering
can be shifted to neural networks  in some cases such as image
classification. Moreover, the availability of cheap computing resources
sped up a shift towards the use of neural networks.

When considering modeling approaches, we examine both fully connected and 
convolutional neural networks. A fully connected neural network is known to 
be inefficient at classifying large images. These inefficiencies in a fully 
connected neural network partly arise because every neuron in the $\ell-1$
layer must be connected to the $\ell$th layer. This requires a series
of matrix multiplications and additions across each layer $\ell$.
Alternatively, a CNN allows for more efficient computations by: (1) efficient
and automatic feature generation using local connectivity and parameter
sharing of convolution operations, (2) dimensionality reduction using pooling
layers. In this project, we attempt to understand these topics in depth
from an empirical stanpoint in relation to image classification using
Google's ``Quick, Draw!'' dataset.


\section{Methods}

\subsection{EDA}


\section{Results}


\subsection{Image Classification using Humans}

\subsection{Fully Connected Neural Network}

\subsection{Convolutional Neural Network}

\section{Discussion}




\bibliographystyle{unsrt}
\bibliography{references}

\end{document}
