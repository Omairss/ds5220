%
% DS 5220 Final
%
\documentclass[12pt]{article}

%
% Packages
%
\usepackage{amsmath}
\usepackage{amssymb}
\usepackage{amsthm}
\usepackage{clrscode3e}
\usepackage{enumerate}
\usepackage{graphicx}
\usepackage[table]{xcolor}% http://ctan.org/pkg/xcolor
\usepackage{caption}
\usepackage{subcaption}


%
% Package settings
%

%% begin.rcode setup, include=FALSE
% library(knitr)
%
% opts_chunk$set(fig.path='figure/latex-', cache.path='cache/latex-')
%% end.rcode

%
% Document Settings
%
\setlength{\parskip}{1pc}
\setlength{\parindent}{0pt}
\setlength{\topmargin}{-3pc}
\setlength{\textheight}{9.5in}
\setlength{\oddsidemargin}{0pc}
\setlength{\evensidemargin}{0pc}
\setlength{\textwidth}{6.5in}


\title{Quick, Draw! Image Classification}
\author{Group 3}

%
% Image directory location
%
\graphicspath{ {./imgs/} }

\begin{document}

\maketitle

\section{Introduction}

Many recent industry innovations have included neural networks. An
important research area within neural networks has been image classification.
One of the best techniques to classify images is Convolutional Neural
Networks (CNN). We have sought to better understand industry best practices
by exploring different image classification techniques with neural networks.
Our exploration starts by manually performing image classification and
noting the level of human accuracy. We then use a fully connected neural
network to perform image classification before implementing a CNN. We
implemented options for tuning and debugging the neural network while
implementing each method during our process. We were able to develop an
intuition for image classification best practices using neural networks
after our research.

Many supervised machine learning methods often differ from those employed 
when constructing neural networks. These supervised machine learning
methods, such as logistic regression or decision tree classification,
require feature engineering to successfully classify an image. Models
requiring feature engineering rely on considerable effort by subject matter
experts to achieve reasonable accuracy. Neural networks allow for an
approach which does not require feature engineering to achieve similar or
better accuracy. Industry engineering efforts regarding feature engineering
can be shifted to neural networks  in some cases such as image
classification. Moreover, the availability of cheap computing resources
sped up a shift towards the use of neural networks.

When considering modeling approaches, we examine both fully connected and 
convolutional neural networks. A fully connected neural network is known to 
be inefficient at classifying large images. These inefficiencies in a fully 
connected neural network partly arise because every neuron in the $\ell-1$
layer must be connected to the $\ell$th layer. This requires a series
of matrix multiplications and additions across each layer $\ell$.
Alternatively, a CNN allows for more efficient computations by: (1) efficient
and automatic feature generation using local connectivity and parameter
sharing of convolution operations, (2) dimensionality reduction using pooling
layers. In this project, we attempt to understand these topics in depth
from an empirical stanpoint in relation to image classification using
Google's ``Quick, Draw!'' dataset.

\section{Methods}

Our methodology consists of reviewing statistical methodology, presenting
theoretically informed hypotheses, and providing an overview of our data.

\subsection{Overview of Statistical Modeling Techniques}

We use human classification, fully connected neural networks,
and a Convolutional Neural Network (CNN). We discuss the CNN in
more detail than was covered in class so an overview is provided
here.

\subsubsection{Convolutional Neural Network (CNN)}

TODO: explain how we used a CNN or anything else not covered in class

\subsection{Hypotheses}

We have generated the following hypotheses based on theoretical
expectations:

\subsubsection{Addressing model component of performance}

\begin{itemize}
  \item[$H1_a$] Fully Connected Neural Network will struggle to get reasonable
    accuracy.
  \item[$H2_a$] CNN should perform better than fully connected neural network.
  \end{itemize}

\subsubsection{Addressing data component of performance}

\begin{itemize}
\item[$H3_a$] Small amounts of data will lead to overfitting.
\item[$H4_a$] Imbalanced class prediction can be improved using data
  augmentation.
\item[$H5_a$] Addition of a category to the trained model will reduce the
  performance of the model.
\item[$H6_a$] Adding more training data will improve the performance of the
  model.
\end{itemize}

TODO: more context would be helpful here, maybe some citations

\subsection{Data}

Google originally collected the data from users by explicitly asking users to
draw objects like forks, hotdogs, etc. The original dataset is 70 GB in size
but Google also provides a simplified version of the dataset- consisting only
of the final images. We used these '.npy' files to train all of our models.
These numpy bitmap files consist of more than 100,000 rows each and 784
features, since they are basically $28 \times 28$ images. Samples of these
images are shown in Figure \ref{fig:quickImages}. The subset of this
dataset sampled varies from experiment to experiment. More details can be
found in the experiments mentioned below.

\begin{figure}[h]
  \begin{center}
    \includegraphics[scale=0.5]{fig1}
  \end{center}
  \caption{Sample images from Quick, Draw! dataset}
  \label{fig:quickImages}
\end{figure}

\section{Results}

We report our results by examining the validity of each hypothesis.

\subsection{Addressing model component of performance}

In order to perform this experiment we built a fully connected neural network
using tensorflow. Our dataset sample selected for this experiment had 10
categories: fish, fork, hotdog, flamingo, airplane, alarm clock, baseball bat,
bicycle, dolphin, elephant. We had 100,000 examples in total, with 10,000
example for each of the aforementioned categories. Our train:test split was
chosen to be 80:20, an industry standard. 

Our initial objective was to build a model that overfit our given dataset,
with the idea that once our model is complex enough to overfit the training
data, we could regularize it in order to achieve a more optimal fit.

After experimenting with various architectures, we narrowed down to a 3 layer
model as specified in Figure \ref{fig:fcnn}.

\begin{figure}[h]
  \begin{center}
    \includegraphics[scale=0.75]{fig2}
    \end{center}
  \caption{Fully Connected Neural Network Architecture}
  \label{fig:fcnn}
  \end{figure}

The training set accuracy observed was 93\%, while the test accuracy was at
73\%, which clearly indicated an overfit in the model, as designed.

The model took 1500 epochs to converge taking around ~45 minutes on a
machine with 13 GB of RAM, Intel Xeon CPU 2.3 GHz (1 core, 2 threads),
1xTesla K80 GPU having 2496 CUDA cores with 12 GB GDDR5 VRAM. In order to
manage variance, we attempted regularizing using dropouts. Unfortunately,
that effort was futile. We then doubled the amount of data sampled to train
the model, eventually helping us achieve training set accuracy of 86\% and
test set accuracy of 82\%. Our learning rate was specified as 0.001.

While our initial hypothesis assumed that a regular fully connected network
would struggle to achieve a reasonable accuracy, our experiment proved our
hypothesis as invalid by achieving a commendable accuracy. Having said that,
the model did take a longer to converge than expected.

As for the CNN, we started off with the same approach with attempting to
overfit the model using convolution operations and a fully connected layer.
Since our objective is to categorise 10 categories, softmax was chosen as the
activation function for the final layer, while every other neuron had ReLU as
the activation function owing to ReLU’s faster convergence.

\begin{figure}
  \begin{center}
    \includegraphics[scale=0.5]{fig3}
    \end{center}
  \caption{CNN Architecture}
  \label{fig:cnnArch}
\end{figure}

\subsubsection{Understanding the Loss Function}

Most modern neural networks are trained using maximum likelihood. This
means that the cost function is the negative log-likelihood, also
described as the cross-entropy between the training data and the model
distribution \cite{Goodfellow-et-al-2016}. In our use case, we are interested
in correctly classifying 10 categories of drawings. Our output unit
use a Softmax function in our output layer because they are most often
used for classifiers representing the probability distribution over
$n$ different classes. Given features $h$, weights $W$, and bias $b$, a
linear layer first predicts unnormalized log probabilities:
\begin{equation}
  z = W^Th + b 
\end{equation}

where $z_i = \log \tilde{P}(y = i | x)$. The softmax function can then
exponentiate and normalize $z$ to obtain the desired $\hat{y}$. The
softmax function is given by
\begin{equation}
  \func{softmax}(z)_i = \frac{\exp(z_i)}{\sum_j \exp(z_j)}.
\end{equation}

As a cost function, we wish to maximize
$\log P(y=i; z) = \log \func{softmax}(z)_i$. We can do this formally as
\begin{equation}
  \log \func{softmax}(z)_i = z_i - \log \sum_j \exp(z_j).
\end{equation}

which gives us the intuition for a single example. Overall, we want to learn
parameters that drive the softmax to predict the fraction of counts of each
outcome observed in the training set \cite{Goodfellow-et-al-2016}:
\begin{equation}
  \func{softmax}(z(x;\theta))_i \approx \frac{\sum_{j=1}^m 1_{y^{(j)} = i,
      x^{(j)} = x}}{\sum_{j=1}^m 1_{x^{(j)} = x}}.
\end{equation}

This is how the $\func{categorical\_crossetropy}(\text{y\_true}, \text{y\_pred})$ function is constructed in Keras \cite{chollet2015keras} as well as the
$\func{softmax\_cross\_entropy\_with\_logits\_v2}(\text{labels}, \text{logits})$
in Tensorflow \cite{tensorflow2015-whitepaper}. 

\subsubsection{Understanding the Optimization Technique}

There are a number of different optimization methods such as Stochastic
and Batch Gradient descent. In practice, we chose to use a faster variant
of gradient descent called Adam. The Adam optimization is able to use
an exponentially weighted average of the gradients, a technique called
momentum, and then use that gradient to update your weights instead. We
also want to slow down gradient descent in the horizontal direction, a
technique called RMSprop), to reduce oscillations from batch gradient descent.
The Adam optimization method is combining momentum and RMSprop to perform
gradient descent more quickly \cite{DeepLear51:online}.


Number of Trainable Parameters: 320,890


The number of clearly being much greater than the number of observations used
to train the model, we need to reduce the number of parameters vis-a-vis
number of observations. From the available strategies to do so, we
incorporated MaxPooling [TODO: why?] post each convolution operation, in
order to extract only the most significant features. As as a consequence,
this reduced the volume of the output, giving the model a smaller memory
footprint as well as a smaller number of trainable parameters.

\begin{figure}
  \begin{center}
    \includegraphics[scale=0.5]{fig4}
  \end{center}
  \caption{CNN Architecture with Reduced Parameters}
  \label{fig:cnnArchPool}
\end{figure}

The observable training accuracy is 0.90, whereas the validation accuracy is
0.89. The model converged after 10 epochs, which took ~ 3 minutes.

Some of the other unsuccessful strategies used to decrease the number of
parameters and to regularize the network were strides and dropouts which
sacrificed way too much in the way of accuracy and thus had to be dropped.
Since most of our drawings had most of their information in the center of the
image, padding, in theory wouldn’t be of much help. However, actually
implementing padding improved our accuracy by 1\% at a cost of ~6000
parameters.

Conclusion

A slight (~5\%) improvement in the accuracy of a CNN when compared to a fully connected network, keeping the number of parameters constant, was observed. However, the CNN happened to be around 13 times faster in practice. This delta in accuracy and speed is likely to be amplified as we scale the model up with respect to model complexity or data.

\subsection{Addressing data component of performance}

ok

\section{Discussion}

Talk about our final slide and tie the hypotheses together



\bibliographystyle{unsrt}
\bibliography{references}

\end{document}
